%
% Copyright (C) 2001 by Holger Karl,
% karl@ft.ee.tu-berlin.de
%
% file: template.tex
%
% Time-stamp: Sat Oct 06 08:29:52 2001
%
% Template fuer die Ausarbeitungen zu TKN-Seminaren
%
\documentclass[12pt,twoside,doublepage]{article}


% Hier den Namen des Teilnehmers und den Titel  der Ausarbeitung eintragen:
\newcommand{\teilnehmer}{Deeksha Mysore Ramesh}
\newcommand{\ausarbeitung}{Computer Networks: Current Advances in Network Softwarization - Literature Survey - Machine Learning and MAC 2 }



% Falls die Ausarbeitung in Deutsch erfolgt,
% die folgenden Kommentar-Zeichen '%' entfernen, andernfalls diese
 % Kommandos auskommentiert lassen:
% Languages:
% \usepackage[german]{babel}
% \usepackage[T1]{fontenc}
% \usepackage[latin1]{inputenc}
% \selectlanguage{german}

%%%%%%%%%%%%%%%%%%%%%%%%%%%
% Im restlichen Vorspann KEINE Aenderungen machen!
%%%%%%%%%%%%%%%%%%%%%%%%%%%
\usepackage{times}
\usepackage{url}

\usepackage{geometry}
\geometry{a4paper,body={5.8in,9in}}

% Graphics:
\usepackage{graphicx}
% aller Bilder werden im Unterverzeichnis figures gesucht:
\graphicspath{{figures/}}

% Headers:
\usepackage{fancyhdr}
% \pagestyle{fancy}
\pagestyle{fancy}
\fancyhead{}
\fancyhead[LE]{ \slshape \teilnehmer}
\fancyhead[LO]{}
\fancyhead[RE]{}
\fancyhead[RO]{ \slshape \ausarbeitung}
\fancyfoot[C]{}

\begin{document}

\title{\ausarbeitung}
\author{\teilnehmer}
\maketitle
\thispagestyle{empty}

%%%%%%%%%%%%%%%%%%%%%%%%%%%%%%%%%%%%%%
% ab hier steht der eigentliche Text:


% Abstract gives a brief summary of the main points of a paper:


% the actual content, usually separated over a number of sections
% each section is assigned a label, in order to be able to put a
% crossreference to it

\section{Intended Outcome of the Seminar}
\label{sec:introduction}

The introduction describes the problem, why it is important, the main
ideas of the following paper, what are the main contributions of the
paper, etc. 

\section{Suggested Work}
\begin{itemize}
	\item \textbf{Deep-Reinforcement Learning Multiple Access for Heterogeneous Wireless Networks - July 2018 \cite{yu2019deep}}
\end{itemize}
 \paragraph{} The paper provides an insights on MAC protocol using the deep-reinforcement learning of Q-learning to accommodate different wireless protocols like TDMA and ALOHA to share a spectrum, using the machine learning strategies.

\label{sec:suggwork}

\section{Related Work}
\label{sec:relwork}
\begin{itemize}
	\item \textbf{Human-level control through deep reinforcement learning - Feb 2015 \cite{mnih2015human}}
	
\end{itemize}
This paper provides detailed information regarding the deep-reinforcement learning and its general application in networks.

\begin{itemize}
	\item \textbf{Reinforcement learning based MAC protocol for wireless sensor networks - July 2006 }
\end{itemize}
\paragraph{}

\section{Model description}
\label{sec:model}

\bibliography{bib}
\bibliographystyle{plain}

\end{document}



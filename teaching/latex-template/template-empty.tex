%
% Copyright (C) 2001 by Holger Karl,
% karl@ft.ee.tu-berlin.de
%
% file: template.tex
%
% Time-stamp: Sat Oct 06 08:29:52 2001
%
% Template fuer die Ausarbeitungen zu TKN-Seminaren
%
\documentclass[12pt,twoside,doublepage]{article}


% Hier den Namen des Teilnehmers und den Titel  der Ausarbeitung eintragen:
\newcommand{\teilnehmer}{Holger Karl}
\newcommand{\ausarbeitung}{IP over Avian Carriers --- A Brief
  Overview}


% Falls die Ausarbeitung in Deutsch erfolgt,
% die folgenden Kommentar-Zeichen '%' entfernen, andernfalls diese
 % Kommandos auskommentiert lassen:
% Languages:
% \usepackage[german]{babel}
% \usepackage[T1]{fontenc}
% \usepackage[latin1]{inputenc}
% \selectlanguage{german}

%%%%%%%%%%%%%%%%%%%%%%%%%%%
% Im restlichen Vorspann KEINE Aenderungen machen!
%%%%%%%%%%%%%%%%%%%%%%%%%%%
\usepackage{times}
\usepackage{url}

\usepackage{geometry}
\geometry{a4paper,body={5.8in,9in}}

% Graphics:
\usepackage{graphicx}
% aller Bilder werden im Unterverzeichnis figures gesucht:
\graphicspath{{figures/}}

% Headers:
\usepackage{fancyhdr}
% \pagestyle{fancy}
\pagestyle{fancy}
\fancyhead{}
\fancyhead[LE]{ \slshape \teilnehmer}
\fancyhead[LO]{}
\fancyhead[RE]{}
\fancyhead[RO]{ \slshape \ausarbeitung}
\fancyfoot[C]{}

\begin{document}

\title{\ausarbeitung}
\author{\teilnehmer}
\maketitle
\thispagestyle{empty}

%%%%%%%%%%%%%%%%%%%%%%%%%%%%%%%%%%%%%%
% ab hier steht der eigentliche Text:

% Abstract gives a brief summary of the main points of a paper:
\begin{abstract}

\end{abstract}

%%%%%%%%%%%%%%%%%%%%%%%%%%%%%%%%%%%%%%
% hier werden - zum Ende des Textes - die bibliographischen Referenzen
% eingebunden
%
% Insbesondere stehen die eigentlichen Informationen in der Datei
% ``bib.bib''
%

\bibliography{bib}
\bibliographystyle{plain}

\end{document}



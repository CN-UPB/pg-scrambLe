\chapter{Hierarchical orchestration}
\label{ch:Hierarchical Orchestration}

\paragraph{} The orchestration of a network service in a MANO is one of the responsibilities of NFVO. The VNMFs along with NFVOs are responsible managing the VNF instances, and its lifecyle. To achieve such end-to-end provisioning, a single MANO platform is not feasible in most of the cases. So, a multi-domain NFVI comes into picture , that allows interaction of multiple administrative domains at different levels. To satisfy the requirements of multi-vendor, multi-technology interoperability environments, the orchestration of network services in a multi-domain can be in a hierarchical fashion or in a peer-to-peer fashion \cite{munoz2018hierarchical}.

\paragraph{} In a peer-to-peer orchestrating model, the MANOs are interconnected to each other in an arbitrary fashion to provide end-to-end network services. his model is preferred when
there is no cross-domain control or cross-domain visibility is required, thus limiting the scaling
possibilities of a MANO \cite{munoz2018hierarchical}.


\paragraph{}In a hierarchical orchestrating model, one of the MANO acts as a global orchestrator to the
underlying MANO with a parent/child hierarchy. This architecture uses a common API between
the parent MANO and child MANOs, hence maintaining a good overview of the global system. Each hierarchical level deals with a minimum of one orchestrator. The orchestrator at any given level is managed by the higher level orchestrator. Some of the factors influencing hierarchical orchestration are the geographical spread and placement of MANOs across regions, scalability constraints of peer-to-peer architectures, different administrative requirements from the operators, with different layers of abstraction. This architecture is preferred because of its broader scope of network services from the lower MANOs, also for the better abstraction. The number of levels of hierarchy in this type of orchestration are based on the geographical region and network domain. The hierarchical levels are also based on abstraction required for the provisioning of network services \cite{munoz2018hierarchical}.

\paragraph{} The below proposed is one of a hierarchical architecting model, to improve scalability for the hierarchical placement system referred in \ref{ch:Scalability Approaches}. 

\paragraph{}The overall orchestration of the model is split on the based of its geographical domains as execution zones (EZ). Execution zones are the representation of physical resources in which services can be deployed. The higher level orchestrations has the limited visibility of the capacity of EZ of the splitted sub domains. The underlying orchestrator of the sub-domain takes care of the placement of services into sub domains, based on (       ) . The lower level orchestrators have limited knowledge about the geographical distributions. These orchestrators also map information from the higher level orchestrators\cite{maini2016hierarchical}.

\paragraph{}The splitting of EZ here is based on the geographical area the domain covers. The grouping of few data centers which are geographically close , for one execution zone. The number of data centers in a region is proportional to the number of execution zones required for the region. Supposedly, here, the whole world is split into 11 execution zones based on their geographical regions like, Western Europe (EUW), Eastern Europe (EUE), Central Asia (ASC), Southern Asia (ASS), Pacific Asia (ASP), Africa (AFR), Northern North America (NAN), Southern North America (NAS), Eastern South America (SAE), Western South America (SAW) and Oceania (OCE). Considering one of the regions, like Western Europe (EUW), it is further split into lower levels based on the resolution domains (RD). The lower levels can further be split based on the granularity of orchestration required\cite{maini2016hierarchical}.

\paragraph{}Each Rds constitutes number of EZs located at a specific location for users. Here, as an example , the first level(level 0)  split is based on 3 resolution domains, RD1, RD2, RD3. With a limited visibility of EZs for the high level orchestrator makes a placement decision . The Rds are divided based on local-local requests of a EZ, remote-local requests of EZ, and local-remote requests of EZ. The next level of split is influenced by the scalable property of orchestrators in first level split\cite{maini2016hierarchical}.
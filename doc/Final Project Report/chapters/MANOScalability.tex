\chapter{MANO Scalability}
\label{ch:Scalability}

In this chapter we discuss the two directions we explored to investigate MANO orchestrator scalability. First, we discuss the scalability plugin that was added to pishahang. The scalability plugin adds 3 main functionalities to pishahang, 1) spawn new child instances of pishahang by allocating new physical resources, 2) Redirecting requests from parent MANO to the child instances and 3) managing the state of child instances. Second, we discuss the experiments conducted on OSM and Pishahang to understand the resource utilization. We also propose a more generic framework to characterize and analyze MANO under load.

\section{Scalability Plugin}

\subsection{Introduction}
\subsection{Architecture}
\subsection{Workflow}




\section{Experiments}
One other task under the MANO scalability investigation was to observe the resource utilization in OSM and Pishahang. To do this we used our own python-mano wrappers to instantiate multiple requests at a time. The next step was to decide on the number of service requests to instantiate. We could simply instantiate any number (like 1,000 or 5000) of service requests but we had to make sure all those requests get successfully instantiated. For this, the infrastructure was the only crucial factor. We used a 16 core virtual machine for installing VIM (Openstack). The recommended ratio of physical CPU to virtual CPU is 1:16, we have 16, thus 16 x 16 = 256 cores \footnote{https://docs.openstack.org/arch-design/design-compute/design-compute-overcommit.html}. Hence we first assigned 1 CPU to one service request.\\

We instantiated 256 service requests on OSM. They were not successfully instantiated. Next, we came up with another parameter called Requests Per Minute(RPM). The first attempt was setting the RPM to be 60 and instantiating 256 service requests. This run was not successful. The next run was 250 service requests at 60 RPM. The requests failed to get instantiated. The third run was 200 instances at 30 RPM, which was also a failure. The idea was to find the right combination of number of service requests and RPM. Finally 180 requests at 30 RPM was found to be suitable for our experiment. This trial and error was done in OSM. Once we got the appropriate requests-RPM combination, the experiment was conducted both in OSM and Pishahang.\\

Next, we had to decide which VNF image to use for our experiment. We used a basic cirros VNF, we instantiated 180 such service requests. We designed their NSDs to have a single VNFD. 

Apart from this, we also wanted to capture the CPU utilization throughout the experiment. We call this lifecycle graphs. The lifecycle of the experiment involves on-boarding, instantiating and terminating network services. This was conducted both on OSM and Pishahang.  


\subsection{Testbed}

This subsection describes the experimental setup that was required for observing resource utilization in scalability analysis.
\subsubsection{Infrastructure}

We used 5 servers, two were installed with OSM and pishahang. The other two were installed with two OpenStack versions and the remaining was installed with kubernetes. OSM was connected to one OpenStack Pishahang was connected to kubernetes and OpenStack The servers had the following machine configuration:
\begin{itemize}
	\item \textbf{OSM (server 1)\\} Intel(R) Xeon(R) CPU E5-2695 v3 @ 2.30GHz with 8 core CPU and 64 GB memory
	
	\item \textbf{Pishahang (server 2)\\} Intel(R) Xeon(R) CPU E5-2695 v3 @ 2.30GHz with 8 core CPU and 64 GB memory
	\textbf{\item Openstack (server 3)\\} Intel(R) Xeon(R) CPU E5-2695 v3 @ 2.30GHz with 16 core CPU and 128 GB memory
	
	\item \textbf{Openstack (server 4)\\} Intel(R) Xeon(R) CPU E5-2695 v3 @ 2.30GHz with 16 core CPU and 128 GB memory
	
	\item \textbf{Kubernetes (server 5)\\}  Intel(R) Xeon(R) CPU E5-2695 v3 @ 2.30GHz with 16 core CPU and 100 GB memory
	


\end{itemize}

\subsubsection{Experiment script}

We used a python script to instantiate 180 requests of a cirros image continuously at 30 RPM with the help of python-mano-wrappers. Once all the requests were instantiated, we once again used the wrappers to send the termination requests all at once. We conducted the same experiment three times i.e we had three runs of this experiment to see the variance. This experiment was conducted both on OSM and pishahang.The results of the experiment is discussed in the following sections.


\subsection{OSM Results}
\todo[inline]{OSM Just like you did in the presentation, explain the functionalities of the top 5 dockers and why they are taking}

\subsubsection{CPU}

\begin{figure}[h]
	\centering
	\includegraphics[width=0.7\linewidth]{../figures/scalability_graphs/Horizontal-Docker-Graphs/osm/cirros_case1_180-CPU}
	\caption{OSM CPU}
	\label{fig:cirroscase1180-cpu}
\end{figure}

\subsubsection{Memory}

\begin{figure}[h]
	\centering
	\includegraphics[width=0.7\linewidth]{../figures/scalability_graphs/Horizontal-Docker-Graphs/osm/cirros_case1_180-MEM}
	\caption{OSM MEM}
	\label{fig:cirroscase1180-mem}
\end{figure}

\subsubsection{Lifecycle}

\begin{figure}[h]
	\centering
	\includegraphics[width=0.7\linewidth]{figures/scalability_graphs/Lifecycle-Graphs-Top-3/OSM-TOP-3-Lifecycle}
	\caption{OSM LS}
	\label{fig:osm-top-3-lifecycle}
\end{figure}


\subsection{Pishahang Results}
The figures \ref{fig:pishcirroscase1180-cpu} \ref{fig:pishcirroscase1180-mem} and \ref{fig:pishahang-top-3-lifecycle} shows the CPU utilization, memory utilization and CPU utilization through out the life cycle of Pishahang dockers during the experiment respectively. Let us consider the important dockers in each case and list their functionalities to realize why they have consumed maximum CPU.


\subsubsection{CPU}

The figure \ref{fig:cirroscase1180-cpu} shows CPU utilization. The first 6 Pishahang dockers are:

\begin{itemize}
	\item \textbf{son-sp-infrabstract:} This docker plays the role of an abstraction layer between the MANO framework and the underlying virtualization infrastructure. It exposes the interfaces to manage services and the VNF instances of all of these 180 instances by reserving resources for their service deployment. It also receives monitoring information about the infrastructure status. Hence, it occupies much of the CPU.
	\item \textbf{sevicelifecyclemanagement:} This is responsible for orchestrating the entire lifecycle of every service that is being deployed with Pishahang service platform. The lifecycle operations of NS include
	
	\begin{itemize}
		\item \textit{NS operations:} 
		1) On-board Network Service
		2) Instantiate Network Service
		3) Scale Network Service
		4) Update Network Service by supporting Network Service configuration changes
		Create, delete, query, and update of VNFFGs associated to a Network Service.
		5) Terminate Network Services
		
	\end{itemize}

	\item \textbf{son-broker:} The Pishahang service platform consists of micro services that use a message broker to communicate, building a flexible orchestration system.
	
The load on all of the other microservices in Pishahang service platform is distributed and the difference between their mean usage is negligible. Hence, the order of all other microservices can be insignificant.

	
	\item \textbf{specificmanageregistry:} The role of this docker is to manage lifecycle of FSM (function-specific manager) and SSM(service-specific manager). Lifecycle operations of FSM and SSM include instantiation, registration, updation and termination. ex: to obtain onboarding SSM request from SLM.
	\item \textbf{cloudservicelifecyclemanagement:}This docker is responsible for lifecycle management of cloud network services on kubernates.
	\item \textbf{functionallifecyclemanagement:}  This docker manages the lifecycle events of each VNF in these 180 network instances. 

\begin{itemize}
		
	\item \textit{VNF operations:} 1) Instantiate VNF (create a VNF using the VNF on-boarding artefacts)
	2) Scale VNF (increase or reduce the capacity of the VNF).
	3) Update and/or Upgrade VNF (support VNF software and/or configuration changes of various complexity).
	4) Terminate VNF (release VNF-associated NFVI resources and return it to NFVI resource pool)
\end{itemize}
	\end{itemize}

\begin{figure}[h]
	\centering
	\includegraphics[width=1\linewidth]{../figures/scalability_graphs/Horizontal-Docker-Graphs/pishahang/cirros_case1_180-CPU}
	\caption{Pishahang CPU}
	\label{fig:pishcirroscase1180-cpu}
\end{figure}
\pagebreak
\subsubsection{Memory Utilization}
The first 5 Pishahang dockers in memory utilization graph  from figure \ref{fig:pishcirroscase1180-mem} are:

\begin{itemize}
	\item \textbf{son-keyclock:} This docker service provides access management and identity management of these micro services.
	\item \textbf{son-monitor-influxdb:} This monitoring plugin records metrics on the internal runtime and service performance and writes it to database.
	\item \textbf{son-sp-infrabstract:}  This docker plays the role of an abstraction layer between the MANO framework and the underlying virtualization infrastructure. It exposes the interfaces to manage services and the VNF instances of all of these 180 instances by reserving resources for their service deployment. It also receives monitoring information about the infrastructure status. Hence, it occupies much of the memory.
	
	\item \textbf{WIM adaptor:} The role of this microservice is to provide connectivity over the physical network. The WIM adaptor acts as a north-bound interface to the higher layers,T eg., NFVO to provide connectivity services between NFVI-POPs or to physical network functions. This also invokes the underlying NFVI network southbound interfaces, whether they are network controllers or NFs, to construct the service within the domain.
	
	\item \textbf{son-gtksrv:} This is Pishahang's gatekeeper service management micro-service.
	
\end{itemize}
 The most important inference from \ref{fig:cirroscase1180-mem} and \ref{fig:pishcirroscase1180-mem} is that, the memory utilization by Pishahang is much lighter than OSM for the top docker services.



\begin{figure}[h]
	\centering
	\includegraphics[width=1\linewidth]{../figures/scalability_graphs/Horizontal-Docker-Graphs/pishahang/cirros_case1_180-MEM}
	\caption{Pishahang MEM}
	\label{fig:pishcirroscase1180-mem}
\end{figure}
\pagebreak

\subsubsection{CPU usage during the lifecycle of this experiment}

We now have the life cycle graphs of the entire experiment. The figure \ref{fig:pishahang-top-3-lifecycle} shows the distribution of the CPU usage among the top 3 dockers throughout the experiment. The experiment lasted for about 10 minutes.

Initially, the metrics of docker containers are recorded for one minute, which is visualized in  the figure \ref{fig:pishahang-top-3-lifecycle} with almost no significant change in the CPU occupancy. The Pishahang infrastructure was reserved for all the 180 instances and with termination requests over the time, the CPU usage was reduced like before. This is observed by the graph of son-sp-infrabstract docker. The graph also depicts the CPU occupancy of other top docker services throughout the experiment.


\begin{figure}[h]
	\centering
	\includegraphics[width=1\linewidth]{figures/scalability_graphs/Lifecycle-Graphs-Top-3/Pishahang-TOP-3-Lifecycle}
	\caption{Pish LS}
	\label{fig:pishahang-top-3-lifecycle}
\end{figure}



\subsection{Limitations of the experiment}

\begin{itemize}
	\item The VM in which we installed the VIM(OpenStack) was supposed to use a configuration with only a 16 core CPU. 
	
	\item Unlike for cirros image, there were issues with successfully instantiating 90 and 180 service requests of Ubuntu image despite multiple runs because Ubuntu image occupied more memory than the cirros image
	
	\item Pishahang doesn't have stable support for VM orchestration.
	
	\item No service function chaining or forwarding graph support in MANO frameworks yet
	
\end{itemize}



\section{MANO Benchmarking Framework} 

\subsection{Introduction}

MANO Benchmarking Framework (MBF) is a result of a small script that was used to run the experiments discussed in the previous sections. The idea of MBF is to provide MANO developers with a generic framework for running experiments on MANO. MBF mainly provides the following 1) Easy interfacing with MANO instances by using python-mano-wrappers, 2) Ability to run experiments with different service descriptors, 3) Collection of performance metrics in convenient data format and 4) Flexible graphing mechanism of the collected data. 

\subsection{Design}

MBF is designed for ease of use and low barrier to entry for developers. We explain the choice of tools that are used in MBF in the following list.

\begin{itemize}
	\item{\textbf{Netdata\footnote{https://github.com/netdata/netdata}}} is the metrics monitoring system for MBF. Netdata captures relevant system metrics and provide powerful APIs to query the recorded data in a suitable format.
	\item{\textbf{Python}} as the choice of scripting language was obvious as the MANOs itself are implemented in python.
	\item{\textbf{python-mano-wrappers}} is used to provide access to REST APIs of MANOs from python.
	\item{\textbf{Docker}} is used to containerize MBF, thus making it easy to distribute and portable.
	\item{\textbf{Matplotlib}} is the graphing library for MBF due to its flexibility and ease of use.
	\item{\textbf{Flask}} as a small python server that can be used to provide additional interactions with the experiment runner.

\end{itemize}

 

\subsection{Parameters and KPIs} 
\todo[inline]{Explain the variable parameters and possible KPIs from the experiments}

\subsection{Steps for the automated experiment run} 
\todo[inline]{A walkthrough of the experiment runner}

\subsection{Example Use Cases}
\todo[inline]{Explain results acquired from MANO benchmarking tool}

\subsubsection{Comparison of different network services} 

Comparison of different network services

\begin{figure}[h]
	\centering
	\includegraphics[width=0.7\linewidth]{figures/scalability_graphs/Docker-Grouped-Cases/osm/osm_lcm-Mean-CPU-Cases}
	\caption{}
	\label{fig:osmlcm-mean-cpu-cases}
\end{figure}

\begin{figure}[h]
	\centering
	\includegraphics[width=0.7\linewidth]{figures/scalability_graphs/Docker-Grouped-Cases/osm/osm_ro-Mean-CPU-Cases}
	\caption{}
	\label{fig:osmro-mean-cpu-cases}
\end{figure}


\subsubsection{Container vs VM Orchestration} 

Container vs VM Orchestration

\begin{figure}[h]
	\centering
	\includegraphics[width=0.7\linewidth]{figures/scalability_graphs/Comparison-VM-Docker/System_metrics_comparison}
	\caption{}
	\label{fig:systemmetricscomparison}
\end{figure}

\begin{figure}[h]
	\centering
	\includegraphics[width=0.7\linewidth]{figures/scalability_graphs/Comparison-VM-Docker/Time_comparison}
	\caption{}
	\label{fig:timecomparison}
\end{figure}


\subsubsection{Scaling Plugin Evaluation}

\begin{figure}[h]
	\centering
	\includegraphics[width=0.7\linewidth]{figures/scalability_graphs/Scalability-Evaluation/Child-TOP-3-Lifecycle}
	\caption{Child scaling}
	\label{fig:child-top-3-lifecycle}
\end{figure}

\begin{figure}[h]
	\centering
	\includegraphics[width=0.7\linewidth]{figures/scalability_graphs/Scalability-Evaluation/Parent-TOP-3-Lifecycle}
	\caption{Parent Scaling}
	\label{fig:parent-top-3-lifecycle}
\end{figure}


\subsection{Future scope}

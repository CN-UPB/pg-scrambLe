\newpage
\section{Splitter}
\subsection{Pishahang}
\begin{table}[htbp] \cite{SONATASchemaDocumentation}
	\begin{center}
		\caption{Schema Parameters of Pishahang Considered for Splitting}
		\label{tab:table1}
		\begin{tabular}{|l|p{0.8\linewidth}|}
			\textbf{Parameter} & \textbf{Description}\\
			
			\hline
			\textbf{descriptor\_version} & The version of the description definition used to describe the network descriptor. \\ 
			\textbf{vendor } & The vendor id allows to identify a VNF descriptor uniquely across all function descriptor vendors.  \\
			\textbf{name} & The name of the network service description. \\
			\textbf{version} & The version of the service descriptor. \\
			\textbf{author} & The person or organization that created the NS descriptor. \\
			\textbf{description} & A longer description of the network service. \\
			\textbf{networkFunctions} & The VNFs (their descriptors), that are part of this network service. \\
			\textbf{connectionPoints} & The connection points of the overall NS, that connects the NS to the external world. \\
			\textbf{virtualLinks} & One to One link between two VNFs. \\
			\textbf{forwardingGraphs} & The forwarding graph describing the topology of the network. \\
		\end{tabular}
	\end{center}
\end{table}
\pagebreak
\subsection{OSM}
\begin{table}[htbp] \cite{OSMSchemaDocumentation}
	\begin{center}
		\caption{Schema Parameters of OSM Considered for Splitting}
		\label{tab:table2}
			\begin{tabular}{|l|p{0.7\linewidth}|}
			\textbf{Parameter} & \textbf{Description} \\
			\hline
			\textbf{\_id} & Identifier for the NSD. \\ 
			\textbf{name} & NSD name. \\
			\textbf{short\_name} & Short name to appear as label on the UI. \\
			\textbf{description} & Description of the NSD. \\
			\textbf{vendor} & Vendor of the NSD. \\
			\textbf{version} & Version of the NSD. \\
			\textbf{connection\_point} & List for external connection points.
Each NS has one or more external connection
points. As the name implies that external
connection points are used for connecting
the NS to other NS or to external networks.
Each NS exposes these connection points to
the orchestrator. The orchestrator can
construct network service chains by
connecting the connection points between
different NS. \\
			\textbf{logo} & File path for the vendor specific logo. \\
			\textbf{ConstituentVnfd} & List of VNFDs that are part of this
network service. \\
			\textbf{scaling \_group \_descriptor} & scaling group descriptor within this network service.
The scaling group defines a group of VNFs,
and the ratio of VNFs in the network service
that is used as target for scaling action. \\
			\textbf{vnffgd} & List of VNF Forwarding Graph Descriptors (VNFFGD). \\
			\textbf{vld} & Virtual Link Descriptor. \\
			\textbf{ip \_profiles} & List of IP Profiles.
  IP Profile describes the IP characteristics for the Virtual-Link. \\
		\end{tabular}
	\end{center}
\end{table}

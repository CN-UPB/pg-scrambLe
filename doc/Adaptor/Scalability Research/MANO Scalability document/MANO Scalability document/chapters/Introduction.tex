\chapter{Introduction}
\label{ch:Introduction}

Scalability in the recent times has become one of the most important factors of the cloud environments. This document explains what scalability is and also gives a few insights about the effects of scaling a system and investigates some scaling approaches that could be incorporated to scale a MANO so as to implement MANO scalablity.

\section{Definition of scaling}
`Scalability' can be defined in different ways:
\begin{itemize}
	 

\item It can be defined as \cite{furht_handbook_2010}"the ability of a particular system to fit a problem as the scope of that problem increases (number of elements or objects, growing volumes of work and/or being susceptible to enlargement)."

\item Also can be defined as \cite{lee_software_2010} "Scalability of service is a desirable property of a service which provides an ability to handle growing amounts of  service loads without suffering significant degradation in relevant quality attributes. The scalability enhanced by scalability assuring schemes such as adding various resources should be proportional to the cost to apply the schemes." 

\item Another definition states that \cite{chieu_scalability_2011} "Scalability is the ability of an application to be scaled up to meet demand through replication and distribution of requests across a pool or farm of servers."

\item According to \cite{noauthor_scale_nodate}
"A system is said to be scalable if it can handle the addition of users and resources without suffering a noticeable loss of performance or increase in administrative complexity"

\end{itemize}

\section{why do we need scaling?}
\paragraph{}
In recent years, there is a large increase seen in the number of users and the resources using the distributed systems.To accommodate the large service requests without loss in performance or increase in administrative complexity, scaling has become an increasingly important factor \cite{ord1994scale}.


\section{System load}
\paragraph{}In a distributed system, the system load is the large amount of data that are to be managed by network services increasing the total number of requests for service.
The load on a MANO can be defined in terms of it's load on NFV Orchestrator to process large number of tasks like on-boarding VNFs. The NFVO of a MANO also receives monitoring information which inturn increases the load on NFVO triggering it to scale the network service across multiple MANOs in a distributed system \cite{soenen2017optimising}.

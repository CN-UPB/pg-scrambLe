\chapter{Capacity of NFVOs and VNFMs}
\label{ch:Capacity of NFVOs and VNFMs}

\paragraph{}In distributed NFVI environment or when network services are spanned over large geographical areas, the number of VNF instances are likely to increase. In a MANO, during the increase on the system load, the number of requests for a service increases, and in turn the number of VNF instances. Hence, there should be adequate number of VNFMs and NFVOs to manage the VNF instances. To determine the capacity of NFVO and VNFM, the Integer Linear Programming (ILP) formula is proposed \cite{abu2017nfv}.

\paragraph{}The below formula determines the optimal number of NFVOs and VNFMs required in a distributed system \cite{abu2017nfv}.



\begin{figure} [H]
	\centering
	\includegraphics[width=0.7\linewidth]{"figures/system model"}
	{\caption*{}}
	\label{}
\end{figure}

\begin{figure}[H]
	\centering
	\includegraphics[width=0.7\linewidth]{"figures/Decision variables"}
	{\caption*{}}
	\label{}
\end{figure}

\begin{figure} [H]
	\centering
	\includegraphics[width=0.7\linewidth]{"figures/math model 1"}
	{\caption*{}}
	\label{}
\end{figure}

\begin{figure}[H]
	\centering
	\includegraphics[width=0.7\linewidth]{"figures/math model 2"}
	{\caption*{}}
	\label{}
\end{figure}

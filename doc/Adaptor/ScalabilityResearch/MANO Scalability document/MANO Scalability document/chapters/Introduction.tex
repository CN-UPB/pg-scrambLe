\chapter{Introduction}
\label{ch:Introduction}

Scalability in the recent times has become one of the most important factors of the cloud environments. This document explains what scalability is and also gives a few insights about the effects of scaling a system and investigates some scaling approaches that could be incorporated to scale a MANO so as to implement MANO scalablity.

\section{Definition of scaling}
`Scalability' can be defined in different ways:
\begin{itemize}
	 

\item It can be defined as \cite{furht_handbook_2010}"the ability of a particular system to fit a problem as the scope of that problem increases (number of elements or objects, growing volumes of work and/or being susceptible to enlargement)."

\item Also can be defined as \cite{lee_software_2010} "Scalability of service is a desirable property of a service which provides an ability to handle growing amounts of service loads without suffering significant degradation in relevant quality attributes. The scalability enhanced by scalability assuring schemes such as adding various resources should be proportional to the cost to apply the schemes." 

\item Another definition states that \cite{chieu_scalability_2011} "Scalability is the ability of an application to be scaled up to meet demand through replication and distribution of requests across a pool or farm of servers."

\item According to \cite{noauthor_scale_nodate}
"A system is said to be scalable if it can handle the addition of users and resources without suffering a noticeable loss of performance or increase in administrative complexity"

\end{itemize}

\section{Why does a MANO need scaling?}
\paragraph{}
In recent years, distributed systems have gained an increase in the number of users and resources. Scaling such a system is an important aspect when large user requests have to be served without compromising system performance or increase in administrative complexity. In terms of MANO, when there are a large number Network Service(NS) instantiations of various network functions, they need to be instatiated considering all the metrics to evaluate an ideal system. The metrics for scalaing are discussed in the section 



\section{System load}
\paragraph{}In a distributed system, the system load is the large amount of data that are to be managed by network services increasing the total number of requests for service.
The load on a MANO can be defined in terms of it's load on NFV Orchestrator to process large number of tasks like on-boarding VNFs. The NFVO of a MANO also receives monitoring information which inturn increases the load on NFVO triggering it to scale the network service across multiple MANOs in a distributed system \cite{soenen2017optimising}.


\section{Metrics to assess a scalable system}
\label{Metrics}
\paragraph{}The definitions of scalability are already discussed earlier in this section. In this section, a few metrics in terms of a MANO server are introduced

\begin{itemize}
	\item \textbf{Speedup} Speedup measures how the  rate of doing work increases with the number of processors, compared to one processor, and has an ideal linear speedup value. \cite{jogalekar_evaluating_2000}
	\item  \textbf{Response time}: Service response time of a MANO is a time period from when a service invocation message is arrived to a MANO on the provider side to when a response for the invocation is returned to the service consumer.
	\item \textbf {Throughput}: It is a metric which measures the efficiency of a MANO to handle service invocations within a given time.
	\item \textbf{Cost}: High scalability under high service loads is an expensive affair. There is always some additional cost involved in planning a scalbility strategy.
	\item \textbf{Performance}: MANO should be able to handle the growing amount of service loads. Scalability should take into account MANOs' ability to manage high service loads without deteriorating Qos.
	\item \textbf{Fault tolerance}: This refers to the ability of a MANO to continue operating without interruption when it's components fail
\end{itemize}s
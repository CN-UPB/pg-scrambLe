\chapter{Non-Functional Requirements}
\label{ch:Non-Functional Requirements}

The important non-functional requirements for the work packages are defined here. Below is the overview of some requirements and description of their relevance in this project.


\begin{enumerate}
	\item \textit {\textbf{Scalability:}} One of the main requirements of any cloud-based networking, is its ability to adapt its throughput to varying load. Here, the load of particular network functions, load on the translator, splitter is also considered as an add on to the overall load of the system. One of the main work packages mentioned in this project is the scalability support for the MANO framework. The adapter plays a major role in scaling up the defined system.
	
	\item \textit {\textbf{Reliability:}} The requirement of reliability aims at the ability of the system to provide continuous correct service. It is the continuation of the service in compliance with the service specification. It is one of the weak requirements that has to be met at any case. In this project, this can also be viewed as providing service for the required duration and then terminating.
	
	\item \textit {\textbf{Availability:}} This term can be paraphrased as \textit {readiness for correct service}, i.e., delivery of service in compliance with the service specification. Here, the work packages should be available for translation or splitting of the NSDs.
	
	\item \textit{\textbf{Flexibility:}} This term defines the ability to accommodate the changing requirements. Here in this project, the system needs to accommodate the MANO framework and its instances. 
	
	\item \textit{\textbf{Performance:}} This term summarizes the possible aspects of the system. In this project, throughput and latency of all the network functions involved is considered as a measurement of efficiency. 
\end{enumerate}


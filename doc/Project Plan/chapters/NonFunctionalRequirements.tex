\chapter{Non Functional Requirements}
\label{ch:Non Functional Requirements}

The important non-functional requirements for our work packages are defined here. Below is the overview over some requirements and describe their meaning for our project in particular.


\begin{enumerate}
	\item \textit {Scalability}: One of the main requirements of any cloud-based networking, is its ability to adapt its throughput to varying load. Here, the load of particular network functions, load on the translator, splitter is also considered as an add on to the overall load of the system. One of the main work packages mentioned in this project is the scalability support for the MANO framework. The adapter plays a major role in scaling up the defined system.
	
	\item \textit {Reliability}: The requirement of reliability aims at the ability of the system to provide continuous correct service. It is the continuation of the service in compliance with the service specification. It is one of the weak requirements that has to be met at any case. In this project, this can also be viewed as providing service for the required duration and then terminating.
	
	\item \textit {Availability}: This term can be paraphrased as \textit {readiness for correct service}, i.e., delivery of service in compliance with the service specification. Here, the work packages should be available for translation or for splitting of the NSDs.
	
	\item \textit{Flexibility}: This terms defines the ability to accomodate the changing requirements. Here in this project, the system need to accomodate the MANO framework and its instances. 
	
	\item \textit{Performance}: This term summarises the possible aspects of the system. In this network scenario , we will basically consider throughput and latency of all the network functions involved, as a measurement of its efficiency. 
	
	
	
\end{enumerate}


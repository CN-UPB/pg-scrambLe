\chapter{Goals and Use Cases}
\label{ch:Goals and Use Cases}

In this chapter, we discuss the intended goals of the project and list some use cases.

\section{Goals}

\paragraph{}
The goal of the project is to develop a software suite which facilitates interoperability between MANO frameworks, thereby enabling management of a network service across multi-vendor environments. To achieve this, we're dividing the software suite into 3 individual work packages (WP), which are developed in parallel initially and finally be merged. In the following sections, we discuss the individual goals of these work packages in detail.

\subsection{WP1: Service Descriptor Translator (SDT)}
\paragraph{}


\subsection{WP2: Service Descriptor Splitter (SDS)}
\paragraph{}


\subsection{WP3: MANO Scalability Support}
\paragraph{}


\subsection{Integration Of Work Packages}
\paragraph{}


\begin{figure}
	\centering
	\includegraphics[width=0.7\linewidth]{figures/Structure_Updated1}
	\caption{This figure visualizes the structure of the project. }
	\label{fig:structureupdated1}
\end{figure}


\newpage
\section{Use Cases}

\subsection{Cross-MANO Framework Interaction}
\paragraph{}

The MANO frameworks used by every network service provider varies from one another. NSD translation enables the deployment of network services that is in accordance with the intended framework.

For instance: Consider two Network Service Operators using different MANO frameworks. One of them uses Sonata framework \cite{draxler2017sonata} and another operator uses OSM framework \cite{ersue2013etsi}. These frameworks have different NSD schemata(refer \ref{SecSONATA} and \ref{SecOSM}). NSD schemata contains VNFs, virtual links, and VNF forwarding graphs and also describes the deployment of a Network Service. By using a translator, these NSD schemata can be translated to framework-specific schema. With this, operators can deploy and manage Network Services across different MANO implementations.

\subsection{Hierarchical Orchestration}
\paragraph{}
By using the MANO adapter, dynamic instantiation of multiple MANO instances and inter-operability between different MANO frameworks can be achieved. The operator will be able to handle the resources in an efficient manner, as one MANO framework can manage a limited number of service requests, operators can explore options to include additional MANO instances under the existing MANO instance to mitigate the traffic load on a single instance. The resources can be provisioned based on the number of requests. This helps the operator in extending their profitability.
\newpage
\textbf{Actors} : The Network Service Providers who would use features of SCrAMbLE.

\begin{figure} [h]
	\centering
	\includegraphics[width=1.0\linewidth]{figures/use-case}
	\caption{Use Case Diagram}
	\label{fig:use-case}
\end{figure}






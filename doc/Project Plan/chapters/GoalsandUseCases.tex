\chapter{Goals and Use Cases}
\label{ch:Goals and Use Cases}


\section{Goals}

\section{Use Cases}

 
\subsection{Cross MANO Framework interaction}
\paragraph{}

As a matter of fact , the MANO frameworks used by every network service provider varies from one another. 
Network service Descriptor translation enables the deployment of network services that is in accordance with the intended framework

For instance : If an operator uses Sonata framework and another operator uses OSM operator, the NSD schemas for both the frameworks will be different. Using the solutions of translation and splitting, network services can be deployed and orchestrated across MANO implementations.

\subsection{Hierarchical orchestration}
\paragraph{}
By implementing MANO adaptor , dynamic instantiation and inter-operability between different MANO frameworks 
can be achieved. As a result of this goal, operators will be able to scale up and scale down the resources as and when required. Also the operator will be able to handle the resources in an efficient manner. 
When there is a high demand for network service , the operator can explore options to include additional MANO instances to mitigate the traffic load on a single MANO instance.The resources can be provisioned based on the amount of requests. This helps the operator in extending profitability. 


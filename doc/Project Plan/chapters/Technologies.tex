\chapter{Technologies}
\label{ch:Technologies}

\section{Orchestrator}

\subsection{}
\subsection{}
\subsection{}
\subsection{}

\section{Virtualized Infrastructure Manager (VIM)}

VIM is one of the three functional blocks specified in the Network Functions Virtualization Management and Orchestration (NFV-MANO) architecture. VIM is responsible for controlling and managing the NFV infrastructure (NFVI), by provisioning and optimizing the allocation of physical resources to the virtual resources in the NFVI. Performance and error monitoring is also a key role of the VIM. Popular VIMs are discussed in the following sections.


\subsection{OpenStack}


\paragraph{}
OpenStack\footnote{https://www.openstack.org/} is a community-driven open source cloud resource management platform. Compute, storage, and networking resources in a datacenter can be provisioned using application program interfaces (APIs) or web dashboard provided by OpenStack component, for instance, NOVA is a component which can provide access to compute resources, such as virtual machines and containers. Network management is enabled by NEUTRON component which handles the creation and management of a virtual networking infrastructure like switches and routers. SWIFT component provides a storage system. By making use of many such components, OpenStack can deliver complex services by utilizing an underlying pool of resources.

\subsection{Amazon Web Services (AWS)}

\paragraph{}
AWS\footnote{https://aws.amazon.com/} is a cloud computing platform, It offers (1) Infrastructure as a Service (IaaS) -- provides resources that can be utilized for custom applications (2) Platform as a Service (PaaS) -- provides services such as database and email which can be used as individual components for building complex applications (3) Software as a Service (SaaS) -- provides user applications with customization and administrative capabilities that are ready to use. AWS is specially preferred by small companies for the flexibility and ease of use of its cloud infrastructure.

\subsection{Kubernetes}

\paragraph{}
Kubernetes\footnote{https://kubernetes.io/} (K8s) is an open-source platform for automation and management of containerized services, it manages computing, networking, and storage infrastructure. Kubernetes was initially developed by Google and now under Cloud Native Computing Foundation. Kubernetes Architecture consists of (1) Master server components -- it is the control plane of the cluster and act as the gateway for  administrators (2) Node Server Components -- servers which are performing work by using containers are called nodes, they communicate with the master component for instructions to run the workload assigned to them. Kubernetes provides comprehensive APIs which are used to communicate between the components and with the external user.



\chapter{Technologies}
\label{ch:Technologies}

\section{MANO Frameworks}

\paragraph{}
In this section, we have listed a few MANO frameworks and their NSD schemata (list of a few parameters and their description). Among several others, the plan is to set them up locally, deploy network services and support them in the project. This list is subjected to future additions or removals.

\subsection{OpenBaton}
\paragraph{}
Open Baton is an open source implementation of ETSI MANO specification. Its main objective is to develop an extensible and customizable framework capable of orchestrating network services from deferent domains\cite{openBaton}. Following table lists out the important parameters of the Network Service Descriptor\cite{openBatonSchemaDocumentation}. 
    \begin{table}[h]
        \centering
        \begin{tabular}{ |p{4cm}|p{10cm}|}
            \hline
            \textbf{Parameter} & \textbf{Description} \\
            \hline
             
             Name & The name of the Network Service \\
             \hline
             Vendor & The vendor or provider of the Network Service \\
             \hline
             Version & The version of the Network Service (can be any string) \\
             \hline
             vnfd & The list of Virtual Network Functions composing the Network Service (see Virtual Network Function Descriptor for more details) \\
             \hline
             Vld & The list of Virtual Links that are referenced by the VNF Descriptors in order to define network connectivity \\
             \hline
             Vnf\_dependency & The list of dependencies between Virtual Network Functions (Not mandatory, please check the VNF Descriptor page in order to understand more about this topic, as you may not need to specifiy dependencies here if those are already provided in terms of requirements at the VNFD level) \\
             \hline
        \end{tabular}
        \caption{Open Baton NSD parameters}
        \label{tab:OpenBatonSchema}
    \end{table}

\subsection{SONATA}
\paragraph{}
SONATA is one of the NSO research projects which targets two crucial technological challenges in the foreseeable future of 5G networks\cite{SONATA}.

\begin{itemize}
    \item Flexible programmability and 
    \item Deployment optimization of software networks for complex services /applications.
\end{itemize}

SONATA complies with the ETSI NFV-MANO reference architecture. Following tables lists out the important parameters of the Network Service Descriptor \cite{SONATASchemaDocumentation}.
\subsubsection{General Network Service Descriptor Section}
    \begin{table}[h]
    \centering
    \begin{tabular}{ |p{4cm}|p{10cm}|}
        \hline
        \textbf{Parameter} & \textbf{Description} \\
        \hline
         
         Vendor & Identifies the NS uniquely across all NS vendors. \\
         \hline
         Name & Name of the NS without its version. \\
         \hline
         Version & Names the version of the NS descriptor. \\
         \hline
         Author & It’s an optional parameter. It describes the author of the network service descriptor \\
         \hline
         Description & It’s an optional parameter. provides an arbitrary description of the network service. \\
         \hline
    \end{tabular}
    \caption{SONATA: General Network Service Descriptor Section}
    \label{tab:SONATA_general_section}
\end{table}

\subsubsection{Network Functions Section}
    \begin{table}[h]
    \centering
    \begin{tabular}{ |p{4cm}|p{10cm}|}
        \hline
        \textbf{Parameter} & \textbf{Description} \\
        \hline
         
         network\_functions & contains all the VNFs that are handled by the network service. \\
         \hline
         Vnf\_id & represents a unique identifer within the scope of the NSD \\
         \hline
         Vnf\_vendor & as part of the primary key, the vendor parameter identifies the VNFD \\
         \hline
         Vnf\_name & as part of the primary key, the name parameter identifies the VNFD \\
         \hline
         Vnf\_version & as part of the primary key, the version parameter identifies the VNFD.\\
         \hline
         Vnf\_description & It’s an optional parameter. a human-readable description of the VNF.\\
         \hline
    \end{tabular}
    \caption{SONATA: Network Functions Section}
    \label{tab:SONATA_NF_section}
 \end{table}


\subsection{TeNOR}
Developed by T-NOVA project, it’s the NFV Orchestrator platform which is not only responsible for entire lifecycle service management of entire Network Function Virtualization but also optimizing the networking and IT resources usage. NS and VNF descriptors follow the TeNORs data model specifications that are a derived and extended version of the ETSI NSD and VNFD data model \cite{de2018network}. Following table lists out some of the parameters of the descriptors \cite{TeNorSchemaDocumentation}:

    \begin{table}[h]
        \centering
    \begin{tabular}{ |p{4cm}|p{10cm}|}
        \hline
        \textbf{Parameter} & \textbf{Description} \\
        \hline
         
         Id	& Unique ID of the Network Service. \\
         \hline
         Name &	Name of the NS. \\
         \hline
         Vendor	& Identifies the NS uniquely across all NS vendors. \\
         \hline
         Version &	Names the version of the NS descriptor \\
         \hline
         Manifest\_file\_md5 & Exposes MD5 check-sums of the meta-data that the manifest file contains. \\
         \hline
         vnfds & Array of virtual network function descriptors. \\
         \hline
    \end{tabular}
        \caption{TeNOR: Network Service Descriptor}
    \label{tab:TeNOR_NSD_section}
 \end{table}

\subsection{Cloudify}
\paragraph{}
Cloudify is an open source cloud orchestration framework mainly focused on optimization of NFV Orchestration and management. It provides a TOSCA based blueprint which felicitates end to end lifecycle of NFV Orchestration. Cloudify follows MANO reference architecture but not entirely compliant to it\cite{de2018network}. Following are some high-level sections of the Blueprint which describes the network services are installed and configured \cite{CloudifySchemaDocumentation}:


    \begin{table}[h]
        \centering
    \begin{tabular}{ |p{4cm}|p{10cm}|}
        \hline
        \textbf{Parameter} & \textbf{Description} \\
        \hline
         
         type &	The node-type of this node template. \\
         \hline
         properties &	The properties of the node template, matching its node type properties schema. \\
         \hline
         instances &	Instances configuration. (Deprecated. Replaced with capabilities.scalable) \\
         \hline
         interfaces	 & Used for a mapping plugins to interfaces operation, or for specifying inputs for already-mapped node type operations. \\
         \hline
         relationships &	Used for specifying the relationships that this node template has with other node templates. \\
         \hline
         capabilities &	Used for specifying the node template capabilities (Supported since: cloudify\_dsl\_1\_3.) Only the scalable capability is supported. \\
         \hline
    \end{tabular}
    \caption{Cloudify: Node Template}
    \label{tab:Cloudify_node_template}
 \end{table}

\subsection{Open Source MANO}
\paragraph{}
OSM is an open source management and orchestration stack in compliant with ETSI NFV information models. OSM architecture splits between resource orchestrator and service orchestrators \cite{de2018network}. Table \ref{tab:OSM_nsd} lists the parameters of NSD schema. \cite{OSMSchemaDocumentation}
    \begin{table}[h]
        \centering
    \begin{tabular}{ |p{4cm}|p{10cm}|}
        \hline
        \textbf{Parameter} & \textbf{Description} \\
        \hline
         
         id &	Identifier for the NSD. \\
         \hline
         name &	NSD Name \\
         \hline
         Short-name &	Short name which appears on the UI \\
         \hline
         vendor &	Vendor of the NSD \\
         \hline
         logo &	File path for the vendor specific logo. For example, icons/mylogo.png. The logo should be part of the network service. \\
         \hline
         description &	Description of the NSD \\
         \hline
         version &	Version of the NSD. \\
         \hline
    \end{tabular}
        \caption{OSM: Network Service Descriptor}
    \label{tab:OSM_nsd}
 \end{table}


\section{Virtualized Infrastructure Manager (VIM)}

VIM is one of the three functional blocks specified in the Network Functions Virtualization Management and Orchestration (NFV-MANO) architecture. VIM is responsible for controlling and managing the NFV infrastructure (NFVI), by provisioning and optimizing the allocation of physical resources to the virtual resources in the NFVI. Performance and error monitoring is also a key role of the VIM. Popular VIMs are discussed in the following sections.


\subsection{OpenStack}


\paragraph{}
OpenStack\footnote{https://www.openstack.org/} is a community-driven open source cloud resource management platform. Compute, storage, and networking resources in a datacenter can be provisioned using application program interfaces (APIs) or web dashboard provided by OpenStack component, for instance, NOVA is a component which can provide access to compute resources, such as virtual machines and containers. Network management is enabled by NEUTRON component which handles the creation and management of a virtual networking infrastructure like switches and routers. SWIFT component provides a storage system. By making use of many such components, OpenStack can deliver complex services by utilizing an underlying pool of resources.

\subsection{Amazon Web Services (AWS)}

\paragraph{}
AWS\footnote{https://aws.amazon.com/} is a cloud computing platform, It offers (1) Infrastructure as a Service (IaaS) -- provides resources that can be utilized for custom applications (2) Platform as a Service (PaaS) -- provides services such as database and email which can be used as individual components for building complex applications (3) Software as a Service (SaaS) -- provides user applications with customization and administrative capabilities that are ready to use. AWS is specially preferred by small companies for the flexibility and ease of use of its cloud infrastructure.

\subsection{Kubernetes}

\paragraph{}
Kubernetes\footnote{https://kubernetes.io/} (K8s) is an open-source platform for automation and management of containerized services, it manages computing, networking, and storage infrastructure. Kubernetes was initially developed by Google and now under Cloud Native Computing Foundation. Kubernetes Architecture consists of (1) Master server components -- it is the control plane of the cluster and act as the gateway for  administrators (2) Node Server Components -- servers which are performing work by using containers are called nodes, they communicate with the master component for instructions to run the workload assigned to them. Kubernetes provides comprehensive APIs which are used to communicate between the components and with the external user.


